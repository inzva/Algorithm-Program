\documentclass[12pt]{article}
\usepackage[utf8]{inputenc}
\usepackage{lmodern}
\usepackage[T1]{fontenc}
\usepackage{amsmath}
\usepackage{enumitem}
\usepackage{graphicx}
\usepackage{fullpage}
\usepackage{siunitx}
\usepackage{fancyhdr}
\PassOptionsToPackage{hyphens}{url}
\usepackage[hyphens]{url}
\usepackage{color}
\usepackage{enumitem}
\usepackage{textcomp}
\usepackage{geometry}
\usepackage{courier}
\usepackage{listings}
\usepackage{array}
\usepackage{amsthm}
\usepackage{mathdots}
\usepackage{amssymb}
\usepackage{minted}
\usepackage{wrapfig}
\usepackage{titlesec}
\usepackage{parskip}
\usepackage{accents}
\usepackage{gensymb}
\usepackage{indentfirst}
\usepackage{courier}
\usepackage{framed}
\usepackage{etoolbox}
\usepackage{titlesec}
\usepackage{appendix}
\usepackage{mdframed}
\usepackage{verbatim}
\usepackage{xspace}
\usepackage{hyperref}
\AtBeginEnvironment{subappendices}{%
	\section*{Appendix}
	\addcontentsline{toc}{section}{Appendices}
}

%\lstset{language=C++,
%                basicstyle=\ttfamily,
%                keywordstyle=\color{blue}\ttfamily,
%                stringstyle=\color{red}\ttfamily,
%                commentstyle=\color{green}\ttfamily,
%                morecomment=[l][\color{magenta}]{\#}
%}

\definecolor{keywordcolor}{rgb}{0,0,0.45}
\definecolor{stringcolor}{rgb}{0.45,0.45,0.45}
\definecolor{commentcolor}{rgb}{0,0.3,0}

\lstset{
	language=C++,
	basicstyle=\footnotesize\ttfamily,
	numbers=left,
	%numberstyle=\tiny,
	frame=tb,
	columns=fullflexible,
	showstringspaces=false,
	breaklines=true,
	tabsize=4,
	keywordstyle=\color{keywordcolor}\footnotesize\bf\ttfamily,
	stringstyle=\color{stringcolor}\footnotesize\ttfamily,
	commentstyle=\color{commentcolor}\it\sffamily
}
% \lstset{basicstyle=\ttfamily,breaklines=true}
\lstloadlanguages{C++}

%\renewcommand{\familydefault}{\sfdefault}

\addtolength{\parskip}{\baselineskip}  
\newcommand{\urlwofont}[1]{\urlstyle{same}\url{#1}}

\renewcommand{\arraystretch}{0.8}
\renewcommand{\headrulewidth}{0pt}
\renewcommand{\footrulewidth}{0pt}

\newcommand{\imagewidth}{0.8\textwidth}

\lhead{}
\chead{}
\rhead{}
\lfoot{}
\cfoot{\thepage}
\rfoot{}

\geometry{
	top=0.9in,
	inner=0.7in,
	outer=0.7in,
	bottom=0.9in,
	headheight=2ex,
	headsep=1ex,
}
\pagestyle{fancy}
%\fancyhf{}
%\setlength{\headsep}{0.2in}


\fancypagestyle{firststyle}
{
	\chead{}
	\setlength{\headsep}{0.0in}
}
\hypersetup{
	unicode=true,
	colorlinks=true,
	linkcolor=blue,
	citecolor=black,
	filecolor=black,
	urlcolor=blue
}

\begingroup
\makeatletter
\@for\theoremstyle:=definition,remark,plain\do{%
	\expandafter\g@addto@macro\csname th@\theoremstyle\endcsname{%
		\addtolength\thm@preskip\parskip
	}%
}
\endgroup

\newtheorem{thm}{Theorem}[section]
\newtheorem{lemma}{Lemma}[section]
\newtheorem{claim}{Claim}[section]
\newtheorem{proposition}{Proposition}[section]
%\theoremstyle{empty}
\newtheorem*{namedthm}{Theorem}

% indention size
%\setlength{\parindent}{19pt}
\setlength{\parindent}{0pt}

% paragraph spacing
\setlength{\parskip}{1em}

% line spacing
\linespread{1}

%\setcounter{tocdepth}{1}


% Documenting starts here! Please do not change above! 

\newcommand{\mytitle}
{
	\textbf {
		inzva Algorithm Programme 2018-2019\\ \ \\
		Bundle 2 \\ \ \\ 
		Algorithms - 1 \\ \ \\
	}
}

\title{\vspace{-2em}\mytitle\vspace{-0.3em}}

\author{
	\textbf{Editor}\\
	Editor's Name  \\ \ \\ 
	\textbf{Reviewers} \\ 
	Reviewer's Name\\
	Reviewer's Name
}

\date{}
\begin{document}
	
	\begin{figure}
		\centering
		\includegraphics[width=\linewidth/4]{inzva-logo.png}
		\label{fig:inzva}
	\end{figure}
	\maketitle
	
	\cleardoublepage
	\tableofcontents
	\markboth{Table of Contents}{}
	\cleardoublepage
	
	\section{Basics}
	
	\subsection{Listing}
	
	A list example from intro-1 document:
	\begin{itemize}
		\item \texttt{ls} - list files in current directory. Usage: \texttt{ls}
		\item \texttt{cd} - change directory. Usage: \texttt{cd \textasciitilde/Desktop}.
		\item \texttt{mkdir} - make a new directory. Usage: \texttt{mkdir directory\_name}
		\item \texttt{mv} - move command(cut). Usage: \texttt{mv source\_path destination\_path}.
		\item \texttt{cp} - copy command. Usage: \texttt{cp source\_path destination\_path}
		\item \texttt{rm} - remove command. Usage: \texttt{rm file\_path}
	\end{itemize}

	\subsection{Links and References}

	Link to \href{https://inzva.com}{inzva} web page.
	
	"A computer would deserve to be called intelligent if it could deceive a human into believing that it was human."\cite{turing}
	
	\subsection{Pages}

	After this points, we can clear the remaining part of the page with \textbf{cleardoublepage} command
	
	\cleardoublepage
	
	\section{Codes and Math}
	
	\subsection{Codes}
	
		\subsubsection{C++}
		
\begin{minted}[frame=lines,linenos,fontsize=\footnotesize]{c++}
int fibonacci( int n ){

	int result = 1, previous = 1;

	for( int i=2 ; i<=n ; i++ ){
		int tmp = result;
		result += previous;
		previous = tmp;
	}

	return result;
}
\end{minted}
		
		\subsubsection{Python}
		
\begin{minted}[frame=lines,linenos,fontsize=\footnotesize]{python}
class Fraction:

	def __init__(self, numerator, denominator):
		self.numerator, self.denominator = numerator, denominator
	
	def bigFraction(a, b):
	
		if a.numerator * b.denominator > a.denominator * b.numerator:
			return a
	
		return b

a, b = Fraction(15, 20), Fraction(12, 18)  # Create two Fractions in order to compare them
biggest = bigFraction(a, b)

print(biggest.numerator, biggest.denominator)
\end{minted}

	\cleardoublepage
	
	\subsection{Mathematical Formulas}
	
	You can write mathematical formulas between \$ symbols. Examples:
	
	$\frac{f(x+h) - f(x)}{h}$, \ $[2, \sqrt{N}]$, \ $h\sum_{i=1}^{r}i^{2}$, \ $f(x) = x^{\dfrac{3}{5+x}}\cdot(x-20)$
	
	You can use double \$ for formatting:
	
	$$\int_{0}^{2}f(x)dx = (c + 0.2*2 + 12.5*2^2 + 2^3) - (c + 0.2*0 + 12.5*0^2 + 0^3) = 58.4$$
	
	\subsubsection{Functions With Cases}
	
	\begin{align*}
	f(n) &= \begin{cases}
	1 & \text{if $n = 0$ or $n = 1$\,\, } \\
	f(n - 1) + f(n - 2) & \text{if $n > 1$\,\,}
	\end{cases}
	\end{align*}
	
	\begin{thebibliography}{0}
		
		\bibitem{turing}
		"Computing Machinery and Intelligence". Book by Alan Turing, 1950.
	\end{thebibliography}

\end{document}
